\chapter*{\centering Abstract}
\addcontentsline{toc}{chapter}{Abstract}
\thispagestyle{cleanPage}
\addac{IoT} is one of the rapidly growing network technologies which has the potential to serve millions of devices. Such \addac{LSN} requires network management to efficiently serve the end-user applications. Traditional Internet architecture suffers from lack of flexibility in the management of the network. The emergence of \addac{SDN} paradigm provides a flexible architecture for network control and management, in the cost of deploying new hardware by replacing the existing routing infrastructure. Further, the centralized controller architecture of \addac{SDN} makes the network prone to single point failure and creates the performance bottleneck. The scalability and failure prone nature of \addac{SDN} becomes the road block to employ \addac{SDN} in context of \addac{LSN}. The objective of this thesis is to design a distributed scalable \addac{SDN} orchestration framework which is suitable for handling the dynamic nature of\addac{LSN}. Additionally in this thesis we explore the performance improvement of \addac{IoT} applications by utilizing \addac{SDN}. 

Modern \addac{IoT} and hand-held devices are equipped with multiple interfaces. To leverage the bandwidth capacity of multiple interfaces several multi-path transport layer protocols exist which provides bandwidth aggregation. The first contribution in this thesis enhances the performance of \addac{IoT} applications by proposing \addac{SDN} control plane application \paperName{SDN-MPTCP} for \addac{MPTCP}, where \addac{MPTCP} is one of the popular multipath transport protocol. In this work we find that, \addac{MPTCP} performance has a strong correlation with the selected paths, and \addac{SDN} can assist in path selection of \addac{MPTCP}. In the next work, we investigate the \addac{SDN} deployment challenges for \addac{LSN}. As mentioned earlier, \addac{SDN} requires deployment of \addac{SDN} supported hardware, which can increase the \addac{capex} of \addac{SDN}. In this work we utilized \addac{NFV} for development of \paperName{FLIPPER}. \paperName{FLIPPER} enables deployment of \addac{SDN} like network management over existing \addac{COTS} devices of \addac{LSN} by converting them into \addac{PDEP}. \paperName{FLIPPER} provides a scalable, flexible, fail-safe and distributed \addTerm{self-stabilized} architecture. In the next contribution we use \paperName{FLIPPER} to design \paperName{Aloe} orchestration framework which utilizes the \addac{INP} platforms of \addac{LSN} to achieve \addTerm{servicification} of \addac{SDN} control plane. \paperName{Aloe} promises \addTerm{plug-and-play} and \addTerm{zero touch deployment} along with light-weight, fault-tolerant and auto-scalable network management platform for \addac{LSN}. Through exhaustive experimentation over an in-house test bed and \addac{AWS} platform we find that, \paperName{Aloe} can significantly improve performance of various \addac{IoT} applications. During this study we also observed that, various end-user applications targeted for \addac{LSN} require \addac{VNF} based \addac{SFC} depending on the network service access policy. However, dynamic deployment of \addacp{VNF} and traffic steering through those \addacp{VNF} to preserve the \addac{SFC} ordering is difficult in a \addac{LSN} which spans across multiple administrative domain. In the next contribution we propose \paperName{Amalgam} which incorporates \addac{SFC} management with \paperName{Aloe} to ensure scalability and dynamic \addac{SFC}. Based on the NP-hard nature of \addac{VNF} placement problem, \paperName{Amalgam} also proposes a greedy heuristic for \addac{VNF} placement which ensures fast flow initialization and provides performance improvement for short-duration flows. As a whole, this thesis provides auto-scalable, fault-tolerant and short-flow friendly distributed architecture and orchestration framework for \addac{LSN} network management to provide fine-grained network control.