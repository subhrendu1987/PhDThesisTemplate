Since the \addac{LSN} can provide a large number of heterogeneous applications, it requires different network-centric services like \addac{NAT}, \addTerm{proxy}, \addTerm{firewall}. In literature, these services are termed as \addTerm{middlebox} applications. End-user application performance depends on locations and performances of the middlebox applications. Therefore, the management of middlebox applications becomes important in \addac{LSN}. The source/destination oriented routing protocols are insufficient for steering traffic through these middleboxes. The problem intensifies when middleboxes are deployed using virtualized platforms (termed as \addac{VNF}) where locations of middleboxes can change dynamically~\cite{haeffner2015service,panda:eecs-2017-141,8696424}. 

Depending on the type of applications, a single flow may require services from multiple \addacp{VNF}. In such cases, the order of execution is also essential. An ordered set of \addacp{VNF} for a particular flow is known as \addac{SFC}. Since \addac{LSN} can span across multiple administrative domains, the development of a unified, scalable framework for the management of \addacp{SFC} and traffic steering through them is not an easy feat. It isn't easy to design a \addac{VNF} management framework that is scalable and still capable of providing \addac{QoS} requirements of the traffics. To satisfy the \addac{QoS} demand in an \addac{LSN} where the number of short-duration flows is significantly high, scheduling and placement of \addacp{VNF} in the actual devices require quick convergence. Furthermore, the entire framework must comply with the \addTerm{plug-and-play} nature of the \addac{LSN} devices.
