Due to the increase in integrated sensors in smart-phones and other hand-held devices, mobile devices have become one of the essential parts of \addac{LSN} deployment~\cite{el2017smartphone}. Improvement in mobile traffic can significantly improve the quality of \addac{LSN} user application performance. Since, the application layer performance is highly dependent on the transport layer performance. Therefore, we aim to find the issues in transport layer protocols used in mobile devices.

Modern mobile devices are usually equipped with multiple hardware interfaces. Avaialability of multiple interfaces can be exploited by aggregating the available bandwidth at all interfaces. The aggregation of bandwidth can be used to satisfy the ever increasing traffic demand. \addac{MPTCP}~\cite{Nikravesh:2016:IUM:2973750.2973769,doi:10.1080/1206212X.2018.1455020,Han:2015:AMW:2716281.2836090,Deng:2014:WLB:2663716.2663727,Mohan2019IsTG} is a transport layer protocol primarily used in data-center and enterprise networks. Usually the hosts used in data-center and enterprise network are equipped with multiple interfaces. \addac{MPTCP} provides the support for bandwidth aggregation in such cases via concurrent usage of different interfaces by creating multiple sub-sockets. 

\addac{MPTCP} initiates multiple sub-flows via different interfaces to aggregate the bandwidth. However, in a network, the path characteristics (such as bandwidth, delay, loss rate, jitter,
etc.) of the underlying sub-flows can be significantly different and time-varying. This diversity adversely affects \addac{MPTCP} performance. Additionally, the time-varying nature of the path characteristic further compounds the problem as it is difficult to estimate them apriori. The difference in end-to-end path characteristics of each active sub-flow may lead to an increase in out of order delivered segments at the receiver side. This increase in out of order delivery leads to \addac{HOL} blocking at the receiver side~\cite{cao2016receiver}. \addac{HOL} blocking also results in delays and increases packet drops, which increase the number of retransmission timeouts. Currently, \addac{MPTCP} uses \addac{RTT} as a measure of path characteristics. However, in the case of \addac{MPTCP}, one segment and its acknowledgment might follow different paths which leads to unreliable measure of path characteristics. On the other hand, the effect of \addac{MPTCP} congestion control and segment scheduling is discussed in the literature~\cite{peng2016multipath,khalili2013mptcp,oh2015constraint,kheirkhah2016mmptcp} also depends on the path characteristics. This issue can be avoided by modelling the \addac{MPTCP} behaviour based on the available end-to-end semantics which to the best of our knowledge none of the prior works tried. The absence of such formal model becomes necessary for minimizing \addac{HOL} blocking and designing of an intelligent path identification method to increase transport layer performance.

%To mitigate these problems and enhance the performance, we argue that instead of relying on simple \addac{RTT} based mechanism, the \addac{MPTCP} capable network management system requires an intelligent dynamic path management scheme. \addac{SDN}~\cite{kreutz2015software} based network management systems can be a savior in such cases, for \addac{MPTCP} flow management over a data-center or an enterprise network where all the network devices are under a single-window management framework. \addac{SDN} provides a logically centralized view of network topology parameters to the application protocols. In \addac{SDN}, control plane gathers statistics periodically from all its data plane devices to maintain consistency of the network states. This makes it feasible to optimize the end-to-end performance of \addac{MPTCP} by selecting a suitable active set of \addac{MPTCP} sub-flows by periodically monitoring the path characteristics at the \addac{SDN} control plane.

%\addac{SDN} improves network manageability. We aim to use the control and data plane segregation of \addac{SDN} to improve the performance of the overall network. Control plane can be utilized to manage end to end network states in \addac{NIB}. Managing networking states in the \addac{NIB}s can help in efficient path setup for transport layer protocols like \addac{MPTCP}~\cite{ford2011architectural} and as an effect it is expected to improve the \addac{IoT} application performance.

%as a transport layer protocol.  \addac{MPTCP} is an emerging transport layer protocol which incurs performance degradation based on the chosen paths. This section is dedicated to the issues observed in \addac{MPTCP}.








%\redtext{4) Details paragraph: What technical challenges did you have to overcome and what kinds of validation did you perform?}
%Optimizing the performance of \addac{MPTCP} based on path characteristics is non trivial due to lack of mathematical model of \addac{MPTCP} systems. Although, the effect of \addac{MPTCP} congestion control and segment scheduling are discussed in the literature~\cite{peng2016multipath,khalili2013mptcp,oh2015constraint}, to the best of our knowledge none of the existing works tried to model the \addac{MPTCP} based on the available end-to-end semantics. 
% \subsection{Contribution}
% The objective of this work is to develop a sub-flow management solution for \addac{MPTCP} based communication over a data-center or enterprise network, where network devices are interconnected via multiple interfaces and there exists large number of paths between the two devices. We first show the impact of sub-flow management on \addac{MPTCP} performance and argue that the default path manager for \addac{MPTCP} works in a sub-optimal way (\S~\cref{C2:motivation}). Based on the motivational study, we propose an analytical model based on Markov chain to characterize \addac{MPTCP} sub-flow selection (\S~\cref{C2:systemmodel}). %The comparative study of the proposed model with experimental results is provided to validate the analytical model. 
% Based on the theoretical foundation, we propose a \addac{SDN} based active sub-flow selection mechanism to improve end-to-end performance of \addac{MPTCP} (\S~\ref{C2:probdef}). We have studied the effect of sub-flow(s) selection using the Mininet~\cite{lantz2015mininet} based testbed in our lab setup (\S~\cref{C2:simulation}), and we observe that the proposed methodology can boost up \addac{MPTCP} end-to-end performance through proper sub-flow management. 

% The rest of the report is organized as follows. In \cref{mptcp_overview} we have briefly described the \addac{MPTCP} protocol and its working principles. To identify the problem, we provide some simple experimental results in \cref{motivation2} which justifies the need for the dynamic path selection. In \cref{systemmodel2}, we propose the Markov chain based theoretical model for \addac{MPTCP} along with the simulation based model verification results. \cref{probdef2} describes the active sub-flow selection problem and our proposed heuristic for path selection. In \cref{simulation2} we describe our testbed setup and the test results. Finally, in \cref{relatedwork2} we provide the literature survey and \cref{conc2} summary of this work.
%\redtext{5) Assessment paragraph: Assess your results and briefly state the broadly interesting conclusions that these results support. This may only take a couple of sentences. I usually then follow these sentences by an optional overview of the structure of the work with interleaved section callouts.}
%\subsection{Contribution}
%\label{Contrib}
%\subsection{Organization}
%\label{org}
