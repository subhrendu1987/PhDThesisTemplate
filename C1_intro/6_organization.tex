Rest of this thesis is organized as follows. \cref{background} provides background and preliminaries to understand various technical aspects of this thesis. \cref{chapter2} proposes \paperName{SDN-MPTCP} which is an \addac{SDN} oriented framework to improve the performance of \addac{MPTCP}. In \cref{chapter3} we analyze the \addac{capex}-\addac{opex} trade-off and propose \paperName{Flipper} which is a scalable control plane architecture suitable for \addac{LSN}. \cref{chapter4} describes proposed \paperName{Aloe} orchestration framework. \paperName{Aloe} is an extension of proposed \paperName{Flipper} and is capable of handling the dynamic nature of the \addac{LSN}.
 In \cref{chapter5} we analyse the \addac{SFC} management issues to propose \paperName{Amalgam}. \paperName{Amalgam} solves the \addac{VNF} placement and traffic steering problem over multiple administrative domains in \addac{LSN}. Finally, we conclude the thesis and suggest possible future works in \cref{futurework}.
