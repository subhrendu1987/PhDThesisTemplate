Apart from the fault, and partition tolerance and \addac{capex} related issues, the dynamic nature of the \addac{LSN} is also difficult to manage. Due to the rise of \addac{IoT}, rapid proliferation of \addac{LSN} has made the network architecture complicated and challenging to manage for service provisioning and ensuring security to end-users. Simultaneously, with the advancement of edge-computing, \addac{INP}, and \addTerm{platform-as-a-service} technologies, end-users consider the network as a service platform for deployment and execution of myriads of diverse applications dynamically and seamlessly over the network.  Consequently, network management is becoming increasingly difficult in today's world with this intricate service-oriented platform overlay on top of the inherently distributed \addTerm{TCP/IP} network architecture. The cost-effective and logically centralized control plane of \addac{SDN} is useful for monitoring, controlling, and deploying new network services. Nevertheless, managing edge and in-network processing over an \addac{LSN} platform are still challenging even with an \addac{SDN} based architecture~\cite{baktir2017can}.

Primary requirements for supporting edge and \addac{INP} over an \addac{LSN} are as follows: (1) Platform should be agile to support the rapid deployment of applications without incurring additional overhead for \addac{INP}~\cite{vaquero2014finding}. The use of \addac{INP} also ensures the scalability of the system~\cite{chiang2017clarifying}. (2) \addac{INP} often requires dividing a service into multiple microservices and deploying the microservices at different network nodes for reducing application response time with parallel computations~\cite{selimi2017practical}. However, such microservices may need to communicate with each other, and therefore flow-setup delay from the in-network nodes needs to be very low to ensure near real-time processing. (3) Percentage of short-lived flows are high for \addTerm{things} centric \addac{LSN}~\cite{Shafiq2013Dec}. This property of \addac{LSN} requires a reduction of flow-setup delay in the network. (4) Failure rates of unmanaged \addac{LSN} devices are in-general high~\cite{kaiwartya2018virtualization}. Therefore, the system should support a fault-tolerant or fault-resilient architecture to ensure liveness. 

Although \addac{SDN} supported edge computing and \addac{INP} have been widely studied in the literature for the last few years~\cite{galluccio2015sdn,tang2018intelligent,baktir2017can} as a promising technology to solve many of the network management problems associated with \addac{LSN}, they have certain limitations. The logically centralized control plane of \addac{SDN} becomes a bottleneck when each flow initiation requires communication between switches and the controller. The bottleneck scenarios can be avoided by using a distributed \addac{SDN} control plane. In such a case, placement of controllers is vital for the reduction in flow performance of \addac{LSN}, where most of the flows are short-lived. On the other hand, static deployment of controllers is not adequate to provide fault-tolerance to \addac{LSN}, where most of the devices show \addTerm{plug-and-play} nature. Therefore, we found that the design of an \addac{SDN} control plane that reduces control plane bottleneck and caters to \addTerm{plug-and-play} devices of an \addac{INP} framework deployed on top of \addac{LSN} are very much necessary.


%therefore, the performance depends on the switch-controller delay. With a single controller bottleneck, the delay between the switch and the controller increases, which affects the flow-setup performance. 


%As we mentioned earlier that the majority of the flows in an \addac{IoT} network are short-lived flows, the impact of switch-controller delay is more severe on the performance of short-lived flows. To solve this issue, researchers have explored distributed \addac{SDN} architecture with multiple controllers deployed over the network~\cite{7218384}. However, with a distributed \addac{SDN} architecture, the question arises about how many controllers to deploy and where to deploy those controllers. Static controller deployments may not alleviate this problem, as \addac{IoT} networks are mostly dynamic with a plug-and-play deployment of devices. Dynamic controller deployment requires hosting the controller software over commercially-off-the-shelf (COTS) devices and designing methodologies for controller coordination which is a challenging task~\cite{ul2017large}. The problem is escalated with the objective of developing a fault-tolerant or fault-resilient architecture in a network where the majority of the flows are short-lived flows. 
%Internet of things (IoT) consists of sensors, actuators and communication devices for monitoring large scale systems. \addac{IoT} generates diversified unprecedented data in short bursts~\cite{Orrevad:Thesis:2009}. To process the data traditional architecture relies on cloud infrastructures. However, latency sensitive applications must be able to withstand the delay required to process the data. The data processing also increases network overhead. In \cite{Botta2016Mar} the authors have shown that, management of application integrating the devices with the help of cloud infrastructure is a major challange that any \addac{IoT} system faces. To overcome these problems, an \addac{IoT} system capable of providing \addTerm{in-network processing} can improve the performance of the system significantly. In-network processing reduces the cost of managing separate cloud infrastructure and network usage. On the other hand by processing \addac{IoT} data inside the network, the application response time could be optimized. However, design of a scalable architecture for providing in-network processing is a challenging task. The major challenge of \addac{IoT} is to maintain connectivity between several \addac{IoT} devices~\cite{iotChallenge}. So maintaining agility of the system capable of in-network processing is complex. In this context, the term agility is defined as a property of deploying \addTerm{rapid innovation and affordable scaling under a common infrastructure}~\cite{openfog}.

%An agile system requires deployment of custom protocols to satify diversified application demand. Therefore, the network architecture must allow rapid deployment of network protocols. To achieve network agility, some of the existing works proposes \addac{SDN}~\cite{7883928,7954011}. \addac{SDN} uses \addTerm{flow} based programability, where a \addTerm{flow} is defined as a set packets with matching header fields. \addac{SDN} employs data and control plane separation to achieve customization of flow processing. The data plane consists of switches, which are responsible for forwarding data through them based on it's flow table entries. Control plane consists of dedicated devices termed as \addTerm{controller}. Task of control plane include generation of flow tables for individual switches. Communication between switches and controllers are managed via \addTerm{OpenFlow} API~\cite{openflow1.3}. \addac{SDN} control plane is highly programmable, which can fulfill the agility goals for in-network processing architecture. \addac{SDN} controllers can communicate with the switches via both \addTerm{in-band} and \addTerm{out-of-band}communication medium. However, \addac{IoT} devices having low number of interfaces require use of in-band controlling.

%Although, in-band \addac{SDN} control plane provides agility, control plane can reduce scalability of the system. Each flow initiation requires communication between switches and controllers. Thus the performance of the flow is dependent on the communication delay between switches and controllers. The in-band control plane is likely to incur more delay than its out-of-band counterpart. The impact of switch-controller delay is higher for the short lived flows than the longer lived flows. \addac{IoT} systems generate higher percentage of short lived flows~\cite{Shafiq2013Dec}. Therefore, the impact of switch-controller communication is higher in case of \addac{IoT} systems. To maintain \addac{QoS}{quality of service}, each \addac{IoT} system imposes a limit on switch-controller latency. This limit is the primary governing parameter on the the size of the network. So, to increase the size of the network a system must keep the switch controller latency under the acceptable limit. Apart from latency, failures are another problem which require attention in case \addac{IoT} systems as resource constrained \addac{IoT} devices are error prone and link failures are frequent. So, it is a common case that a set of switches looses connectivity with the controller. In such cases the \addac{IoT} application performance gets affected heavily. 

%In contrast to the existing architectures, the novelty of this work is as follows. We integrate an \addac{SDN} control plane with the in-network processing infrastructure, such that the control plane can dynamically be deployed over the \addac{COTS} devices maintaining a fault-tolerant architecture. This has multiple advantages for an \addac{IoT} framework with in-network processing capabilities: (a) The distributed controller approach ensures that there is no performance bottleneck near the controller. (b) The flow-setup delay is significantly minimized because of the availability of a controller near every device. (c) The fault-tolerant controller orchestration ensures the liveness of the system even in the presence of multiple simultaneous devices or network faults. To achieve these goals, we first design a distributed, robust, migration-capable and elastically scalable control plane framework with the help of docker containers~\cite{ma2017efficient} and state-of-the-art control plane technologies. The proposed control plane consists of a set of small controllers, called the micro-controllers, which can coordinate with each other and help in deploying new applications for in-network processing. The container platform helps in installing these micro-controllers on the \addac{COTS} devices; a container with a micro-controller can be seamlessly migrated to another target device if the host device fails, yielding a fault-tolerant architecture. In addition to this, the deployment mechanism for the micro-controllers ensure elastic auto-scaling of the system; the total number of controllers can grow or sink based on the number of active devices in the \addac{IoT} network. We develop a set of special purpose programming interfaces to ensure fault-tolerant elastic auto-scaling of the system along with intra-controller coordinations. Finally, we design a set of \addac{API} over this platform to ensure language-free independent deployment of applications for in-network processing. Combining all these concepts, we present Aloe, a distributed, robust, elastically auto-scalable, platform-independent orchestration framework for edge and in-network processing over \addac{IoT} infrastructures. 

%We have implemented a prototype of Aloe using state-of-the-art \addac{SDN} control plane technologies and deployed the system over an in-house testbed and a $68$-node Amazon web services platform. The in-house testbed consists of $10$ nodes (Raspberry Pi devices) with Raspbian kernel version 8.0. As mentioned, we have utilized docker containers to host the distributed control plane platform. We have tested Aloe with three popular applications for in-network \addac{IoT} data processing -- (a) A web server (simple python based), (b) a distributed database server (\texttt{Cassandra}), and (c) a distributed file storage platform (\texttt{Gluster}).  We observe that Aloe can reduce the flow-setup delay significantly (more than three times) compared to state-of-the-art distributed control plane technologies while boosting up application performance even in the presence of multiple simultaneous faults. 
