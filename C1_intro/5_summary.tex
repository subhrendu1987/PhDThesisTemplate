In this thesis, we propose solutions to the issues mentioned earlier by developing SDN control plane applications and orchestration frameworks. The proposed solutions presented here rely on several aspects of \addac{LSN} like scalability, incremental deployment issues, transport layer protocols, network management systems, and service chaining management. The step-by-step contributions of this thesis are as follows.
\subsection{Improvement of MPTCP Performance}
The first major contribution of our thesis is an intelligent dynamic path management scheme for \addac{MPTCP} traffics that optimizes the traffic performance as mentioned in \cref{C0:Q1}. To develop this path manager, we rely on the \addac{SDN} control plane which provides a logically centralized view of network topology parameters by periodically obtaining statistics from all its data plane devices~\cite{nunes2014survey}. This makes it feasible to optimize the end-to-end performance of \addac{MPTCP} by selecting a suitable active set of \addac{MPTCP} sub-flows. In order to identify a suitable active set, we provide a formal model of \addac{MPTCP} by using an irreducible and aperiodic \addac{DTMC}. The proposed formal model provides an estimation of \addac{MPTCP} throughput and receiver buffer length based on end-to-end path characteristics (latency, available bandwidth, etc.) of the sub-flows. We use this estimation mechanism to develop an \addac{SDN} control plane application named as \paperName{SDN-MPTCP}. The performance of the proposed solution is compared with various baselines. During the evaluation period, we suffered from the lack of real \addac{LSN} experimental facility. This challenge motivated us to investigate the deployment challenges of an \addac{SDN} enabled \addac{LSN} infrastructure.
\subsection{SDN Deployment Over \addac{LSN}}
The primary challenge to design a suitable \addac{SDN} control plane for \addac{LSN} infrastructure is to reduce the\addac{capex} as mentioned in \cref{C0:Q2}. Therefore, in this thesis we design \paperName{Flipper}. \paperName{Flipper} is somewhere in-between traditional architecture and \addac{SDN} based architecture, where \addac{COTS} routers dynamically change their roles from a conventional network router to an \addac{NIB} and participate in \addac{PDEP} functionalities. \paperName{Flipper} reduces \addac{capex} by using \addac{COTS} devices with the help of \addac{NFV}~\cite{koponen2010odc}\footnote{At the time of this research NFV was less popular}. We also propose a distributed self-stabilizing \addac{NIB} placement algorithm which reduces the \addac{opex} by ensuring fault and partition tolerance. We also provide formal proofs to ensure the linear convergence of the proposed algorithm. The performance of \paperName{Flipper} is analyzed from both simulations through a synthetic network environment and real implementation over an emulation platform using \addTerm{network name-space}. Our implementation of Flipper provides proof-of-concept support of the new architecture while comparing performance with existing methods in terms of flow initiation delay.
\subsection{Providing Plug and Play Support}
 We extended the \paperName{Flipper} principles to develop \paperName{Aloe}, which is a fault tolerant \addac{SDN} orchestartion framework for dynamic \addac{INP} platforms. \paperName{Aloe} is custom built to cater the \addTerm{plug-and-play} devices (\cref{C0:Q3}) of \addac{LSN}. \paperName{Aloe} primarily serves two purposes: (a) easy and improved management of \addac{LSN} application generated flows (b) without increasing additional \addac{capex}. To implement these two features, \paperName{Aloe} exploits the capabilities of \addac{INP} platforms and proposes \addTerm{servicification}\footnote{Servicification is defined as ``transformation of existing system into one or more discrete services"~\cite{define:servicification}} of control plane. \paperName{Aloe} ensures auto-scalability which is desired for a large scale network like \addac{LSN}. Additionally, our proposed framework preserves the \paperName{Flipper} properties like fault-tolerance and linear time convergence which helps to reduce the flow initiation time significantly. Thus, we found significant performance improvement for various end user applications in our experimental set-up using an in-house test bed and a large scale \addac{AWS} platform. 
\subsection{Enhancing Capability of SDN Managed \addac{LSN} Using ``middlebox" Application Management}
\paperName{Aloe} is further extended into \paperName{Amalgam} to combat the issues given in \cref{C0:Q4}. \paperName{Amalgam} couples distributed \addac{SFC} management and SDN enabled traffic steering framework. \paperName{Amalgam} can extend its services over multiple administrative domains by exploiting in-network processing~\cite{10.1007/978-3-030-19759-9_6,225992} architecture. \paperName{Amalgam} ensures fine-grained \addac{QoS}. Moreover, \paperName{Amalgam} is compatible to cater to the \addTerm{plug-and-play} nature of the devices without compromising operation, where, the plug-and-play devices may join and leave the platform dynamically. The coupling of \addac{VNF} placement and traffic steering in \paperName{Amalgam} ensures dynamic service chaining during an on-going session. To evaluate performance of \paperName{Amalgam} we develop an emulation framework \emph{MiniDockNet} for \addac{VNF} deployment using \addTerm{docker}~\cite{docker} over in-network processing, as the existing network name-space oriented mininet~\cite{Lantz:2010:NLR:1868447.1868466} emulator is not sufficient for in-network processing. The performance of the proposed framework is compared with some of the existing works, which shows that \paperName{Amalgam} can provide better end-to-end delay than it's predecessors for short-duration flows.
