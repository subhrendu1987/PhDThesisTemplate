\firstOcc{IoT} refers to an interconnecting infrastructure to integrate everyday used embedded computing devices. Recently \addac{IoT} is being used for improving the quality of life~\cite{NOUR201995}. In an \addac{IoT}, the number of end-users in a single network can reach up to a million~\cite{iotdeployment} very easily.  Due to this, the global \addac{M2M} traffic is estimated to reach $51\%$~\cite{ipassonline} of the total traffic demand in $2023$. Therefore, \addac{IoT} is estimated to grow as a major technology in the near future. In this thesis, we focus on \firstOcc{LSN}. Like \addac{IoT}, \addac{LSN} also spans from backbone network to edge devices. We identify \addac{LSN} as a \addac{WAN} which is a subset of \addac{IoT}. We make the following key assumptions to segregate \addac{LSN} from \addac{IoT} in this thesis.
\begin{itemize}
\item The \addac{LSN} contains millions of heterogeneous resource constraint \firstOcc{COTS} devices. Each device can have multiple interfaces. The traffic generated by the \addac{LSN} devices is mainly short-flows~\cite{6560419}.
\item Since it is difficult to deploy such a vast network while maintaining single administrative domains, \addac{LSN} spans across multiple administrative domains.
\item By looking at the momentum of virtualization technologies used presently~\cite{10.1145/3341302.3342075}, we firmly believe that virtualization can be an inherent technology used in the future \addac{LSN}.
\end{itemize}