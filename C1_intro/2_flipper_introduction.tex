Apart from the transport layer performance issue, the biggest challenge in \addac{LSN} is to maintain the scalability of the network. Let us consider the following scenario where the network administrator of an \addac{LSN} wants to dynamically update bandwidth distribution policies based on network usage statistics. The network is connected with multiple network service providers, and therefore she needs to update the configuration at different edge routers and gateways. With traditional network devices, like layer $3$ switches, this task is tedious. Even a minor configuration inconsistency among the edge routers and gateways may lead to severe network under-utilization or bandwidth imbalance. Further, the system is also not scalable for such dynamic updates of network configuration policies.

\addac{SDN}~\cite{kreutz2015software} can help in dynamic network configuration update. \addac{SDN} uses a centralized controller to convert system administrator defined policies to device configurations and apply those configurations in the targeted networking devices. By using the programmable controller, \addac{SDN} separates the network control plane from the data plane. The \addac{SDN} control plane takes care of all the control functionalities (like forwarding decision) based on the network parameters and installs the control decisions to the data plane devices. In contrast, the data plane is only responsible for forwarding packets based on the configuration parameters set by the control plane. 

Although \addac{SDN} has revolutionized dynamic network management aspects, it requires specific hardware that can understand the instructions given by the \addac{SDN} controller. Therefore the critical question is: {\em How much effort and cost does one need to convert an existing network infrastructure to an \addac{SDN} supported one?} An \addac{SDN} supported hardware is much costlier than a \addac{COTS} network device, which requires huge \addac{capex} to replace existing infrastructure by \addac{SDN} supported infrastructure. Deployment of \addac{SDN} supported equipments incrementally can be one way to avoid this extra cost. On the other hand, there are existing \addac{SDN} control plane architectures~\cite{sezer2013we,levin2013incremental,levin2014panopticon} which proposes interoperability between the \addac{SDN} and non-SDN devices. However, in both the cases fine grained network control can be ensured. Therefore, we require a technique that can transform \addac{COTS} device into an \addac{SDN} supported \addac{PDEP} device in order to reduce the cost of deployment. On the other hand, the use of \addac{COTS} devices as \addac{PDEP} can increase the failure rate, which increases \addac{opex}. By ensuring fault and partition tolerance the \addac{opex} can be reduced which motivated us to understand a suitable design of \addac{SDN} that can satisfy the above mentioned challenges of \addac{SDN} deployment over \addac{LSN}.
%%%%%%%%%%%%%%%%%%%%%%%%%%%%%%
%Although it is quite inevitable that the future of network management is \addac{SDN}, simultaneously we also ask this question: {\em Can we make our existing network more management friendly, such that dynamic network configuration becomes possible without much changing the existing infrastructure?} This work tries to find out the answer to this question. We show that it is quite possible to use the existing \addac{COTS} routers to work as network \addac{PDEP}, which are known as \addac{NIB}. Additionally, we find that, 

%However, it is challenging to turn a \addac{COTS} router to a \addac{NIB} by installing a few additional software tools to support \addac{NFV}~\cite{koponen2010odc}.


%We can turn a \addac{COTS} router to a \addac{NIB} by installing a few additional software tools to support \addac{NFV}~\cite{koponen2010odc}. With the help of \addac{NFV} functionalities, a \addac{COTS} router can dynamically update the policy control parameters within its neighborhood~\cite{berde2014onos,foster2011frenetic}. Accordingly, we develop a new network management architecture, which is somewhere in-between the traditional architecture and \addac{SDN} based architecture, where the \addac{COTS} routers dynamically change their roles from a conventional network router to a \addac{NIB} and participate in \addac{PDEP} functionalities. We call this architecture \term{Flipper}. 

%Flipper has two specific advantages over \addac{SDN} based network architecture, among others. First, to implement Flipper, a network administrator does not need to procure new costly hardware and second, Flipper avoids the controller bottleneck problem~\cite{sezer2013we,dixit2013towards,koponen2010odc,yao2014capacitated,phemius2014disco} which is much debated in the \addac{SDN} research community. Flipper is a distributed architecture, where the \addac{COTS} routers execute a distributed self-stabilizing algorithm to decide which nodes can work as a \addac{NIB}. As the \addac{NIB}s have limited resources because they are built on top of the existing routers, a \addac{NIB} can manage, control and update the network policies only among its neighborhood. Therefore, we develop a distributed self-stabilizing \term{MIS} selection mechanism, which is indeed non-trivial. To maintain consistency in policy decisions across the network, we have developed a fault-tolerant \addac{NIB} selection mechanism. We analyse the closure, fault-tolerance and scalability properties of Flipper. The performance of Flipper is analysed from both simulations through a synthetic network environment, as well as through real implementation over an emulation platform using \term{network name-space}. Our implementation of Flipper provides a proof-of-concept support of the new architecture, while compare the performance with that of other protocols in terms of flow initiation delay.


